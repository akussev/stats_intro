\section{Контрольная работа №1: условия}

Следующие задачи предлагались группе Б05-024 на контрольной работе. Задачи 3 и 4 у обоих вариантов совпадают

\begin{center}
    \textbf{\large Левый вариант}
\end{center}
\noindent
\textbf{1.} Пусть $X_1, \ldots, X_n$ -- выборка из распределения с плотностью
\[
\rho_{\theta}(x) = \frac{1 + \theta x}{2} I(-1 \le x \le 1),\;\;\;\theta \in \Theta = [-1;1].
\]
Рассмотрим оценку $\widehat{\theta}(X) = 3 \overline{X}$. Является ли она несмещённой, состоятельной, сильно состоятельной, асимптотически нормальной?

\noindent
\textbf{2.} Дана выборка $X_1, \ldots, X_n$ из распределения Лапласа со сдвигом $\theta > 0$ (плотность равна $\rho_{\theta}(x)=\frac{1}{2}e^{-|x-\theta|}$). Постройте асимптотический доверительный интервал уровня доверия $\gamma$ для параметра $\theta$.

\noindent
\textbf{3.} \textbf{(а)} Пусть обобщённая плотность экспоненциального семейства распределений $\pth$ имеет вид
\[
\rho_{\theta}(x) = g(x) \exp{\left(u(x)\theta - b(\theta)\right)}.
\]
Докажите, что $\me u(X) = \frac{\partial}{\partial \theta} b(\theta)$ и $\va u(X) = \frac{\partial^2}{\partial \theta^2} b(\theta)$.

\textbf{(б)} По выборке $X_1, \ldots, X_n$ из распределения с плотностью
\[
\rho_{\theta}(x) = \frac{\theta e^{-x}}{(1+e^{-x})^{\theta + 1}}
\]
постройте оптимальную оценку для $\tau(\theta) = \frac{1}{\theta^2}$.

\noindent
\textbf{4.} Приведите пример состоятельной, но не асимптотически нормальной оценки

\textit{Замечание.} Мы считаем константу нормально распределённой.

\begin{center}
    \textbf{\large Правый вариант}
\end{center}
\noindent
\textbf{1.} По выборке $X_1, \ldots, X_n$ из распределения с плотностью
\[
\rho_{\theta}(x) = \frac{1}{x\theta \sqrt{2\pi}}\exp{\left(-\frac{\ln^2{x}}{2\theta^2}\right)}I(x > 0)
\]
постройте сильно состоятельную оценку параметра $\theta$.

\noindent
\textbf{2.} Пусть $X_1, \ldots, X_n$ -- выборка из отрицательного биномиального распределения $\mathrm{NB}(M, \theta)$, то есть $\pth(X_i = k) = C_{k+m-1}^{m-1} (1-\theta)^k \theta^m$ для $k\in\{0, 1, \ldots\}$. Параметр $m$ известен. Для каких функций $\tau(\theta)$ существует эффективная оценка? Найдите $i(\theta)$ -- информацию одного наблюдения выборки.

\noindent
\textbf{3.} \textbf{(а)} Пусть обобщённая плотность экспоненциального семейства распределений $\pth$ имеет вид
\[
\rho_{\theta}(x) = g(x) \exp{\left(u(x)\theta - b(\theta)\right)}.
\]
Докажите, что $\me u(X) = \frac{\partial}{\partial \theta} b(\theta)$ и $\va u(X) = \frac{\partial^2}{\partial \theta^2} b(\theta)$.

\textbf{(б)} По выборке $X_1, \ldots, X_n$ из распределения с плотностью
\[
\rho_{\theta}(x) = \frac{\theta e^{-x}}{(1+e^{-x})^{\theta + 1}}
\]
постройте оптимальную оценку для $\tau(\theta) = \frac{1}{\theta^2}$.

\noindent
\textbf{4.} Приведите пример состоятельной, но не асимптотически нормальной оценки

\textit{Замечание.} Мы считаем константу нормально распределённой.


