\section{Пререквизиты}\label{prerequisite}

Для дальнейшей комфортной и плодотворной работы хотелось бы напомнить некоторые основные темы из курса теории вероятностей, которые непременно и многократно пригодятся нам впоследствии. Очевидно, в подобного рода разделе вряд ли удастся пересказать весь необходимый материал, и далее мы будем пользоваться теорией, сильно выходящей за его рамки. Здесь мы не будем даже пояснять, например, что такое математическое ожидание, измеримое отображение и прочее --- не в этом цель. Главное для нас --- подсветить некоторые ключевые моменты сейчас, чтобы не разбирать их на ходу, когда нам приспичит их пустить в ход. Если выражаться метафорически, мы заблаговременно зарядим несколько чеховских ружей, которые будут тут и там выстреливать в течение рассказа. Это отчасти и минус, потому что в данный момент совершенно неясно, зачем мы обсуждаем тот или иной объект из теории вероятностей --- многое покажется нелогичным и непонятно зачем взявшимся. Однако смею Вас заверить, что чтением данных вводных слов зря время потрачено не будет. Итак, приступим.

Отправной точкой в теории вероятности является \textit{вероятностное пространство}: тройка $(\Omega, \mathcal F, \pth[])$, где $\mathcal F$ --- $\sigma$-алгебра на $\Omega$, а $\pth[]$ --- вероятностная мера, заданная на $\mathcal F$. Само по себе оно не представляет особого интереса с практической точки зрения --- это всего лишь математическая модель абстрактного <<чёрного ящика>>, источника случайности. Куда интереснее для нас будет то, что из этого чёрного ящика до нас доносится, а именно случайные величины и векторы. 

Напомним, что \textit{случайным вектором} называется измеримое отображение $\boldxi$ из $(\Omega, \mathcal F)$ в $(\R^k, \mathcal B (\R^k))$, для $k=1$ его обычно называют \textit{случайной величиной}. Измеримость позволяет перенести меру $\pth[]$ на более осязаемое пространство $(\R^k, \mathcal B (\R^k))$ по правилу
$$
\mathsf{P}_{\boldxi}(B) := \pth[](\boldxi \in B), \;\;\; B \in \mathcal B (\R^k);$$
вероятностную меру $\mathsf{P}_{\boldxi}$ называют \textit{распределением случайного вектора $\boldxi$}.

Ввиду своей важности распределения на $\R^k$ хотелось бы как-то легко задавать. Самым популярным способом является \textit{функция распределения}, которая для распределения $\pth[]$ определяется как
\[
F\colon (x_1, \ldots, x_k) \mapsto \pth[]((-\infty; x_1) \times \ldots \times (-\infty; x_k)).
\]

Широко известно, что по значениям данной функции исходная вероятностная мера определяется однозначно. Более того, если функция $F\colon \R \to  [0; 1]$ обладает не слишком обременительными свойства (грубо говоря, монотонность, непрерывность справа и какие надо пределы на бесконечности), то существует распределение с функцией распределения $F$.

\begin{example}
	Для дискретных случайных величин функция распределения представляет собой кусочно-постоянную функцию, претерпевающую разрывы в точках из носителя распределения. Например, для распределения Бернулли $\bernd(p)$, которое определяется как $\pth[](\{1\}) = p$, $\pth[](\{0\}) = 1 - p$, функция распределения имеет вид
	\[
	F(x) = \left\{\begin{aligned}
		&0, \;\; &x < 0;\\
		&1-p,\;\; &x \in [0, 1);\\
		&1, \;\; &x \ge 1.
	\end{aligned}\right.
	\]
	
	Их противоположностью являются распределения с непрерывной функцией распределения, у которых вероятность всякой одной точки равна нулю. Самый простым примером такового является равномерное распределения $\ud[a, b]$, название которого себя полностью оправдыввет --- вероятность всякого подотрезка отрезка $[a, b]$ пропорциональна его длине:
	
	\[
	F(x) = \left\{\begin{aligned}
		&0, \;\; &x < a;\\
		&\frac{x-a}{b-a},\;\; &x \in [a, b);\\
		&1, \;\; &x \ge b.
	\end{aligned}\right.
	\]
\end{example}

Отметим полезное свойство функции распределения в одномерном случае, которое позволяет от любого непрерывного распределения перейти к одному и тому же --- равномерному.

\begin{proposition}\label{f_to_uniform}
    Пусть $\xi$ --- случайная величина с непрерывной функцией распределения $F(x)$. Тогда $F(\xi) \sim \ud(0, 1)$.
\end{proposition}

\begin{proof}
    Для $t \in (0, 1)$ определим $F^{-1}(t)$ как $\sup\{x\colon F(x) = t\}$ (множество, по которому берётся супремум, не пусто в силу непрерывности). Тогда из неравенства $F(\xi) \le t$ следует $\xi \le F^{-1}(t)$. В обратную сторону импликация выполняется почти наверное: вероятность события
    \[
    \inf\{x\colon F(x) = t\} < \xi \le \sup\{x\colon F(x) = t\}
    \]
    равна нулю, так как функция распределения $\xi$ одинакова при левой и правой части сего неравенства. Значит,
    \[
    \pth(F(\xi) \le t) = \pth(\xi \le F^{-1}(t)) = F(F^{-1}(t)) = t,
    \]
    что есть функция распределения $\ud(0, 1)$. 
\end{proof}

В многомерном случае такое уже не сработает (почему?), да и вообще функции распределения в многомерье становятся не совсем тривиальным объектом для изучения и работы. Но благо есть другой, более простой способ задания распределения --- через плотность. Если нам дана неотрицательная борелевская функция $\rho\colon \R^k \to \R_+$ с единичным интегралом по $\R^k$, то на её основе можно задать распределение следующим образом:
\[
\pth[](B) := \int_{B} \rho(\mathbf x) \, d\mathbf x, \;\;\; B \in \mathcal B(\R^k).
\]

Такие распределения ещё называют \textit{абсолютно непрерывными}. Минус такого подхода заключается в его узости --- например, дискретные распределения не задашь через интеграл по классической мере Лебега, потому что какую бы плотность мы ни взяли, её интеграл по носителю дискретного распределения будет нулевым. Что же делать? Очень уж хотелось бы и в таком случае определять распределения как интеграл по какой-то одной конкретной мере. Только по какой?

\begin{definition}
    \textit{Считающей мерой} на $\Z^k$ называется функция $\mu\colon \mathcal{B}(\R^k) \to \N \cup \{+\infty\}$, определённая как
    \[
    \mu(B) = \sum_{\mathbf{x} \in \Z^k} I(\mathbf{x} \in B).
    \]
\end{definition}

Ясно, что это $\sigma$-конечная мера, как и классическая мера Лебега на $\R^k$, и она равна числу целочисленных точек, которое попадает в данное множество, откуда собственно и пошло название. Несложно также понять, как считается интеграл произвольной измеримой функции $f$ по этой мере:
\[
\int_{\R^k} f(\mathbf{x})\,\mu(d\mathbf{x}) = \sum_{\mathbf{x} \in \Z^k} f(\mathbf{x}),
\]
если, конечно, этот ряд сходится абсолютно. Такое представление позволяет записывать матожидание от дискретных случайных векторов $\xi\colon \Omega \to \Z^k$ через интеграл с плотностью по считающей мере:
\[
\me[] g(\xi) = \sum_{\mathbf{x} \in \Z^k} g(\mathbf{x}) \cdot \pth[](\xi = \mathbf{x}) = \int_{\R^k} g(\mathbf{x}) \cdot \rho(\mathbf{x})\,\mu(d\mathbf{x}),
\]
где $\rho(\mathbf{x}) = \pth[](\xi = \mathbf{x})$ --- \textit{плотность по мере $\mu$}. Отныне под словами <<плотность по мере $\mu$>> мы будем подразумевать либо обычную плотность, которая у нас была ранее, по классической мере Лебега, либо дискретную по считающей мере. Семейства распределений, у которых есть плотность по одной и той же мере, мы будем называть \textit{доминируемым}.

{
\footnotesize Данные соображения можно распространить на случай произвольной $\sigma$-конечной меры $\mu$ с помощью теоремы Радона-Никодима.

\begin{definition}
     Пусть $\mu$ --- некоторая $\sigma$-конечная мера на $\R^n$. Семейство вероятностных мер $\mathcal{P}$ называется \textit{доминируемым} относительно меры $\mu$, если $\forall \mathsf{P} \in \mathcal{P}\colon \mathsf{P} \ll \mu$ (напомним, что $\nu \ll \mu$, если $\forall B\in\R^n\colon \mu(B) = 0 \Rightarrow \nu(B) = 0$). Производную Радона-Никодима $\frac{d\mathsf{P}}{d\mu}$ называют \textit{обобщённой плотностью}.
\end{definition}

Несложно показать, что выполнена \textit{формула пересчёта}. Она позволяет перейти от интеграла по неизвестной мере к интегралу плотности по известной мере, которая и доминирует семейство:
\[
\int_{\R^n} f(\mathbf{x})\,\mathsf{P}(d\mathbf{x}) = \int_{\R^n} f(\mathbf{x}) \cdot \frac{d\mathsf{P}}{d\mu}\,\mu(d\mathbf{x}).
\]
}

% TODO обобщённая плотность

% TODO виды сходимостей, их наследование, лемма Слуцкого

% TODO предельный теоремы

% TODO УМО

