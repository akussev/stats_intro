\section{Оценки и их свойства}

В теории вероятности мы в основном работали с заранее известными распределениями: изучали их свойства, вводили их характеристики и занимались прочими вещами. Математическая статистика делает всё с точностью до наоборот: по свойствам выборки надо определить, из какого распределения она пришла. Часто набор распределений, которые являются кандидатами на роль истинного распределения, можно описать набором параметров, поэтому в основном нашей задачей будет определить с некоторой точностью значение параметра по реализации распределения.

Напомним, что мы работаем в вероятностно-статистической модели $(\mathcal{X}, \mathcal{B}(\mathcal{X}), \mathcal{P})$, где $\mathcal{X}$ -- множество результатов наблюдения (обычно под ним мы подразумеваем $\R$ или $\R^n$, но также это может быть и $\R^{\infty}$, когда мы хотим работать с бесконечными выборками), $\mathcal{B}(\mathcal{X})$ -- борелевская $\sigma$-алгебра на $\mathcal{X}$, $\mathcal{P}$ -- семейство вероятностных мер на $\mathcal{B}(\mathcal{X})$ (чаще всего оно будет иметь вид $\{\pth\colon \theta \in \Theta\}$, где $\theta$ -- неизвестный параметр). Так как распределение теперь не фиксировано, то часто в обозначения матожидания, дисперсии, сходимости по вероятности и прочих вещах, которые используют конкретное распределение, мы будем писать индекс $\theta$, чтобы подчеркнуть, какое распределение используется в данный момент.

\begin{definition}
Пусть $(\Omega, \mathcal{F})$ -- измеримое пространство. Произвольная $(\mathcal{B}(\mathcal{X})|\mathcal{F})$-измеримая функция $S\colon \mathcal{X} \to \Omega$ называется \textit{статистикой}.

В случае, когда $\mathcal{P}$ описывается параметром $\theta \in \Theta$, а $\Omega = \Theta$, такая статистика называется \textit{оценкой} параметра $\theta$.
\end{definition}

\begin{remark}
1. Статистика по умолчанию не зависит от неизвестного параметра, иначе получается странно: хотим найти неизвестный параметр с помощью статистики, при этом она почему-то включает в себя этот параметр. Иногда всё же мы будем рассматривать функции, использующие фиксированное значение параметра (например, в построении доверительных интервалов).

2. Обратите внимание, что оценкой мы априори считаем что-то, что имеет значения лишь в допустимом множестве $\Theta$. Какой бы хорошей статистикой $S(X)$ не была, если она периодически выдаёт недопустимое значение параметра, то она неадекватна. Иногда этим свойством мы будем пользоваться, так что имейте это в виду.
\end{remark}

Чтобы понимать, какие оценки хорошие, а какие -- не очень, нужно выделить некоторые свойства оценок, которые было бы крайне желательно иметь.

Первое из них говорит, что если нам будут поступать раз за разом выборки, то оценки для них будут в среднем похожи на истинный параметр:
\begin{definition}
Оценка $\widehat{\theta}(X)$ называется \textit{несмещённой} оценкой параметра $\tau(\theta)$, если для любого $\theta \in \Theta$ $\me \widehat{\theta} = \tau(\theta)$.
\end{definition}
Это свойство весьма логичное, но его очевидно недостаточно, чтобы утверждать, что оценка хоть сколь-нибудь пригодна. Например, если $\me X_1 = \theta$, то $X_1$ -- несмещённая оценка параметра $\theta$, хотя она не использует всю мощь выборки. Это подводит нас к асимптотическим свойствам оценок: нам бы хотелось, чтобы при увеличении размера выборки увеличилась бы и точность в предсказании параметра.
\begin{definition}
Пусть $(X_1, \ldots, X_n, \ldots)$ -- выборка. Оценка $\theta^*_n$ называется \textit{состоятельной} оценкой параметра $\tau(\theta)$, если для любого $\theta \in \Theta$ выполнено $\theta^*_n \stackrel{\pth}{\to} \theta$. Оценка $\theta^*_n$ называется \textit{сильно состоятельной} оценкой параметра $\tau(\theta)$, если для любого $\theta \in \Theta$ выполнено $\theta^*_n \stackrel{\pth\text{-п.н.}}{\longrightarrow} \theta$.
\end{definition}

Вообще под (сильно) состоятельной оценкой подразумевают последовательность оценок, но обычно все и так понимают, о чём речь. Если из контекста понятно, как именно зависит оценка от параметра $n$, то нижний индекс оценки убирают.

Не менее интересной для рассмотрения оказывается скорость сходимости оценки к истинному значению параметра. 
\begin{definition}
Оценка $\theta^*_n$ называется \textit{асимптотически нормальной} оценкой параметра $\tau(\theta)$, если для любого $\theta \in \Theta$ $$\sqrt{n}(\theta^*_n - \theta) \stackrel{d_{\theta}}{\longrightarrow} \mathcal{N}(0, \sigma^2(\theta)).$$ Величина $\sigma^2(\theta)$ называется \textit{асимптотической дисперсией}.
\end{definition}
Так как нормально распределённая случайная величина по "правилу трёх сигм"\, принимает своё значение на интервале $(-3\sigma(\theta); 3\sigma(\theta))$ с очень высокой вероятностью, то можно считать, что оценка стремится к истинному значению параметра со скоростью порядка $3\sigma(\theta)/\sqrt{n}$. 

Заметим, что из сильной состоятельности или асимптотической нормальности оценки следует её состоятельность (по поводу последнего смотрите задачу \ref{an_to_sost}).

\textbf{Методы доказательства асимптотических свойств}:
\begin{enumerate}
    \item ЗБЧ, УЗБЧ, ЦПТ и их многомерные аналоги
    \item Теорема о наследовании сходимостей
    \item \begin{theorem*}[о наследовании асимптотической нормальности]
    Пусть $\widehat{\theta}(X)$ -- асимптотически нормальная оценка с асимптотической дисперсией $\sigma^2(\theta)$. $\tau(\theta)$ -- дифференцируемая на $\Theta$ функция. Тогда 
    \[
    \sqrt{n}(\tau(\widehat{\theta}) - \tau(\theta)) \stackrel{d_{\theta}}{\longrightarrow} \mathcal{N}(0, \sigma^2(\theta)\cdot (\tau'(\theta))^2).
    \]
    \end{theorem*}
    \begin{remark}
    Можно (хотя не рекомендуется) понимать это так: к $\widehat{\theta}$, которая является "примерно"\, нормально распределённой при больших $n$, применяется преобразование, которое в малой окрестности $\theta$ "почти что"\, линейное c коэффициентом $\tau'(\theta)$, отчего и дисперсия увеличилась в $(\tau'(\theta))^2$ раз.
    \end{remark}
\end{enumerate}

\begin{problem}
    Пусть $X_1, \ldots, X_n$ -- выборка из $U(0, \theta)$. Проверьте на несмещённость, состоятельность, сильную состоятельность и асимптотическую нормальность следующие оценки параметра $\theta$: \textbf{(а)} $\overline{X} + X_{(n)}/2$, \textbf{(б)} $(n + 1)X_{(1)}$, \textbf{(в)} $X_{(1)} + X_{(n)}$. \textbf{(г)} Найдите число $\delta$ > 0 и невырожденное распределение $F_{\theta}$ такие, что $n(\theta - X_{(n)}) \stackrel{d_{\theta}}{\longrightarrow} \xi \sim F_{\theta}$
\end{problem}

\begin{solution}
    Для начала поймём, как распределены первая и последняя порядковые статистики. Для $t \in (0; 1)$:
    \begin{gather*}
        \pth(X_{(n)} \le t) = \pth(X_1, \ldots, X_n \le t) = \prod \pth(X_i \le t) = \pth(X_1 \le t)^n = \frac{t^n}{\theta^n};\\
        \rho_{X_{(n)}}(t) = \frac{nt^{n-1}}{\theta^n}I(0 < t < \theta).
    \end{gather*}
    \begin{gather*}
        \pth(X_{(1)} \le t) = 1 - \pth(X_{(1)} > t) = 1 - \pth(X_1, \ldots, X_n > t) = 1 - \prod \pth(X_i > t) =\\
        = 1 - (1 - \pth(X_1 \le t))^n = 1 - \left(1 - \frac{t}{\theta}\right)^n;\;\;\; \rho_{X_{(1)}}(t) = \frac{n}{\theta}\left(1 - \frac{t}{\theta}\right)^{n-1} I(0 < t < \theta).
    \end{gather*}
    Также полезным для проверки на несмещённость будут их матожидания:
    \begin{gather*}
        \me X_{(n)} = \int_0^{\theta} t \cdot \frac{nt^{n-1}}{\theta^n}\,dt = \frac{n}{n+1}\theta,\\
        \me X_{(1)} = \int_0^{\theta} t \cdot \frac{n}{\theta}\left(1 - \frac{t}{\theta}\right)^{n-1}\,dt = n\theta \int_0^1 s(1-s)^{n-1}\,ds = n\theta B(2, n) = n\theta \frac{\Gamma(2)\Gamma(n)}{\Gamma(n+2)} = \frac{\theta}{n+1}.
    \end{gather*}
    Теперь мы готовы к решению задачи.
    
    \textbf{Несмещённость.} Из линейности матожидания легко видеть, что несмещёнными будут $(n + 1)X_{(1)}$ и $X_{(1)} + X_{(n)}$.
    
    \textbf{Состоятельность.} $\overline{X} \stackrel{\pth\text{-п.н.}}{\longrightarrow} \theta/2$ из УЗБЧ. Для произвольного $0 < \epsilon < \theta$ (для $\epsilon > \theta$ всё ясно):
    \[
    \pth(|X_{(n)} - \theta|>\epsilon) = \underbrace{\pth(X_{(n)} > \theta + \epsilon)}_{\phantom{1}=0} + \pth(X_{(n)} < \theta - \epsilon) = \frac{(\theta - \epsilon)^n}{\theta^n} \to 0.
    \]
    Что же насчёт первой порядковой статистики, то
    \[
    \pth(|(n+1)X_{(1)} - \theta|>\epsilon) \ge \pth((n+1)X_{(1)} > \theta + \epsilon) = \left(1 - \frac{\theta + \epsilon}{\theta(n+1)}\right)^n \to \exp\left(-\frac{\theta+\epsilon}{\theta}\right) \ne 0
    \]
    С другой стороны, для
    \[
    \pth(|X_{(1)}|>\epsilon) = \pth(X_{(1)} > \epsilon) = \left(1 - \frac{\epsilon}{\theta}\right)^n \to 0.
    \]
    Таким образом, $X_{(n)} \stackrel{\pth}{\rightarrow} \theta$, $X_{(1)} \stackrel{\pth}{\rightarrow} 0$, но $(n+1)X_{(1)} \stackrel{\pth}{\nrightarrow} \theta$. Из всего этого получаем, что $\overline{X} + X_{(n)}/2$ в силу того факта, что сходимости по вероятности можно складывать, будет состоятельной, $(n + 1)X_{(1)}$ не является состоятельной оценкой $\theta$, а вот $X_{(1)} + X_{(n)}$ уже будет являться как сумма $X_{(n)}$, стремящейся по вероятности к $\theta$, и $X_{(1)}$, стремящейся по вероятности к нулю.
    
    \textbf{Сильная состоятельность.} Тут всё куда проще, ведь при фиксированной выборке что $X_{(n)}$, что $X_{(1)}$ -- монотонны при увеличении $n$, а это значит, что из их сходимости по вероятности будет следовать сходимость $\pth$-п.н. Действительно, как известно из курса теории вероятностей, у последовательности, сходящейся по вероятности, есть подпоследовательность, сходящаяся почти наверное. Тогда из монотонности следует, что и вся последовательность такая. Отсюда, $X_{(n)} \stackrel{\pth\text{-п.н.}}{\longrightarrow} \theta$, $X_{(1)} \stackrel{\pth\text{-п.н.}}{\longrightarrow} 0$, поэтому оценки $\overline{X} + X_{(n)}/2$ и $X_{(1)} + X_{(n)}$ будут сильно состоятельными. $(n + 1)X_{(1)}$ же таковой не является, так как она даже не состоятельна.
    
    \textbf{Асимптотическая нормальность.} Аналогично предыдущему пункту и по задаче \ref{an_to_sost} $(n + 1)X_{(1)}$ не асимптотически нормальна. Проверим, как себя ведут порядковые статистики:
    \begin{gather*}
        \pth(\sqrt{n}(X_{(n)} - \theta) \le t) = \pth\left(X_{(n)} \le \theta + \frac{t}{\sqrt{n}}\right) = 
        \left\{
        \begin{aligned}
        &1,\,\,\,t \ge 0;\\
        &\left(1 + \frac{t}{\theta\sqrt{n}}\right)^n \to 0,\,\,\,t<0.
        \end{aligned}
        \right.\\
        \pth(\sqrt{n}X_{(1)} \le t) = \pth\left(X_{(1)} \le \frac{t}{\sqrt{n}}\right) = 
        \left\{
        \begin{aligned}
        &0,\,\,\,t < 0;\\
        & 1 - \left(1 - \frac{t}{\theta\sqrt{n}}\right)^n \to 1,\,\,\,t>0.
        \end{aligned}
        \right.
    \end{gather*}
    Стало быть, $\sqrt{n}(X_{(n)} - \theta), \sqrt{n}X_{(1)} \stackrel{d_{\theta}}{\longrightarrow} 0 \sim \mathcal{N}(0, 0)$ (мы натуралы, поэтому считаем нуль нормально распределённым), и по лемме Слуцкого $\sqrt{n}(\overline{X} + X_{(n)}/2 - \theta) \stackrel{d_{\theta}}{\longrightarrow} \mathcal{N}(0, \va \overline{X})$, $\sqrt{n}(X_{(1)} + X_{(n)} - \theta) \stackrel{d_{\theta}}{\longrightarrow} 0$. Таким образом, эти оценки будут ещё и асимптотически нормальными.
    
    \textbf{(г)} Подберём $\delta$ так, чтобы распределение $n^{\delta}(\theta - X_{(n)})$ было чем-то нетривиальным.
    \begin{gather*}
        \pth(n^{\delta}(\theta - X_{(n)}) \le t) = \pth(X_{(n)} \ge \theta - t n^{-\delta}) = 
        \left\{
        \begin{aligned}
        &0,\,\,\,t \le 0;\\
        &1 - \left(1 - \frac{t}{\theta n^{\delta}}\right)^n,\,\,\,t > 0.
        \end{aligned}
        \right. \eqto
    \end{gather*}
    Как мы видим, при $\delta < 1$ распределение будет тривиальным, а при $\delta > 1$ и вовсе получается что-то неадекватное. При $\delta = 1$ же:
    \[
    \eqto \left\{
        \begin{aligned}
        &0,\,\,\,t \le 0;\\
        &1 - e^{-\frac{t}{\theta}},\,\,\,t > 0.
        \end{aligned}
        \right.,
    \]
    что есть функция распределения для $Exp\left(\frac{1}{\theta}\right)$.
\end{solution}

\begin{problem}\label{an_to_sost}
    Пусть $\theta^*(X)$ -- асимптотически нормальная оценка параметра $\theta$. Докажите, что тогда $\theta^*(X)$ является состоятельной оценкой $\theta$.
\end{problem}

\begin{solution}
    С одной стороны, по условию $\sqrt{n}(\theta^*-\theta) \stackrel{d_{\theta}}{\longrightarrow} \mathcal{N}(0, \sigma^2(\theta))$. С другой, очевидно выполняется $1 / \sqrt{n} \stackrel{d_{\theta}}{\longrightarrow} 0$. Тогда по лемме Слуцкого $\theta^* - \theta \stackrel{d_{\theta}}{\longrightarrow} 0$. Из сходимости по распределению к константе следует сходимость к ней по вероятности, а значит, $\theta^* \stackrel{\pth}{\longrightarrow} \theta$.
\end{solution}

\begin{problem}
    Пусть $X_1, \ldots, X_n$ -- выборка из некоторого распределения с параметром $\sigma^2$. Пусть, кроме того, $\va X_1 = \sigma^2$. Назовём \textit{выборочной дисперсией} статистику $s^2 = \sum (X_i - \overline{X})^2 / n$ Докажите, что:
    
    \textbf{(а)} $s^2 = \overline{X^2} - \overline{X}^2$;
    
    \textbf{(б)} $s^2$ является сильно состоятельной оценкой для $\sigma^2$;
    
    \textbf{(в)} если $\me X_1^4 < \infty$, то $s^2$ является асимптотически нормальной оценкой для $\sigma^2$.
    
    \textbf{(г)} Является ли она несмещённой оценкой для $\sigma^2$?
\end{problem}

\begin{solution}
    \textbf{(а)} 
    \begin{gather*}
        s^2 = \sum \left( \frac{X^2_i}{n} - \frac{2 X_i \overline{X}}{n} + \frac{\overline{X}^2}{n} \right) = \frac{\sum X^2_i}{n} -  2\overline{X}\frac{\sum X_i}{n} + \overline{X}^2 = \overline{X^2} - 2\overline{X}^2 + \overline{X}^2 = \overline{X^2} - \overline{X}^2.
    \end{gather*}
    \textbf{(б)} По УЗБЧ $\overline{X} \stackrel{\pth\text{-п.н.}}{\longrightarrow} \me X_1$, а $\overline{X^2} \stackrel{\pth\text{-п.н.}}{\longrightarrow} \me X^2_1$. Значит, по теореме о наследовании сходимости почти наверное и предыдущему пункту: $s^2 \stackrel{\pth\text{-п.н.}}{\longrightarrow} \me X^2_1 - (\me X_1)^2 = \va X_1 = \sigma^2$.
    
    \textbf{(в)} По многомерному ЦПТ (её можно применять, так как из конечности $\me X^4_1$ следует конечность вторых моментов у координат вектора):
    \[
    \sqrt{n}\left(
    \begin{pmatrix}
    \overline{X}\\
    \overline{X^2}
    \end{pmatrix}
    -
    \begin{pmatrix}
    \me X_1\\
    \me X_1^2
    \end{pmatrix}
    \right)
    \stackrel{d_{\theta}}{\longrightarrow} \mathcal{N}(0, \Sigma),
    \]
    где $\Sigma$ -- некоторая ковариационная матрица. Применяя теорему о наследовании асимптотической нормальности для $\tau(x, y) = y - x^2$:
    \[
    \sqrt{n}\left(s^2 - \sigma^2\right) = \sqrt{n}\left(\tau(\overline{X}, \overline{X^2}) - \tau(\me X_1, \me X_1^2)\right) \stackrel{d_{\theta}}{\longrightarrow} \mathcal{N}(0, \nabla \tau^T \Sigma\nabla \tau),
    \]
    что и требовалось.
    
    \textbf{(г)} Несмотря на то что оценка $s^2$ обладает такими потрясающими свойствами, она имеет смещение:
    \begin{gather*}
        \me S^2 = \me \left( \frac1n\sum X_i^2 - \frac{1}{n^2}\sum_{i, j} X_i X_j \right) = \frac{1}{n}\sum \me X_i^2 - \frac{1}{n^2} \sum \me X_i^2 - \frac{1}{n^2} \sum_{i \ne j} \me(X_i X_j) =\\
        = \me X_1^2 - \frac{1}{n} \me X_1^2 - \frac{1}{n^2} \underbrace{\sum_{i \ne j} \me X_i \me X_j}_{n^2 - n\text{ слагаемых}} = \frac{n-1}{n} \me X_1^2 - \frac{n-1}{n}(\me X_1)^2 = \frac{n-1}{n} \va X_1 = \frac{n-1}{n}\sigma^2.
    \end{gather*}
    
\end{solution}

\begin{problem}
    Пусть $X_1, \ldots, X_n$ -- выборка из экспоненциального распределения с параметром $\theta$, т.е. $p_{\theta}(t) = \theta e^{-\theta t}I(t > 0)$. Покажите, что для любого $k\in\N$ статистика $\left(k! / \overline{X^k}\right)^{1/k}$ является асимптотически нормальной оценкой параметра $\theta$. Найдите её асимптотическую дисперсию.
\end{problem}

\begin{solution}
    Для начала вспомним, что $\me X_1^k = k! / \theta^k$ (это можно получить, честно найдя интеграл или рассмотрев хар. функцию). По ЦПТ:
    \[
    \sqrt{n}\left(\overline{X^k} - \frac{k!}{\theta^k}\right) \stackrel{d_{\theta}}{\longrightarrow} \mathcal{N}(0, \va X_1^k) = \mathcal{N}(0, \me X_1^{2k} - (\me X_1^k)^2) = \mathcal{N}\left(0, \frac{(2k)!-k!^2}{\theta^{2k}}\right).
    \]
    Применим теорему о наследовании асимптотической нормальности для $\tau(x) = \left(\frac{k!}{x}\right)^{1/k}$, но сначала посчитаем её производную
    \[
    \tau'(x) = - \frac{k!^{1/k}}{k x^{1+1/k}};\;\;\;\tau'\left(\frac{k!}{\theta^k}\right) = \frac{\theta^{k+1}}{k!\cdot k}.
    \]
    Таким образом,
    \[
    \sqrt{n}\left(\left(k! / \overline{X^k}\right)^{1/k} - \theta\right) \stackrel{d_{\theta}}{\longrightarrow} \mathcal{N}\left(0, \frac{(2k)!-k!^2}{\theta^{2k}}\cdot \frac{\theta^{2k+2}}{k!^2\cdot k^2} \right) = \mathcal{N}\left(0, \frac{\theta^2((2k)!-k!^2)}{k!^2\cdot k^2} \right).
    \]
\end{solution}

\begin{problem}
    Пусть $X_1, \ldots, X_n$ -- выборка из распределения $Bern(\theta)$. Предположим, что функция $\tau$ такова, что существует несмещённая оценка для $\tau(\theta)$. Докажите, что $\tau$ является многочленом степени не выше $n$. Любой ли такой многочлен подойдёт?
\end{problem}

\begin{solution}
    Пусть $\widehat{\theta}(X)$ -- несмещённая оценка для $\tau(\theta)$. Тогда можно честно посчитать её матожидание:
    \[
    \tau(\theta) = \me \widehat{\theta}(X) = \sum_{\mathbf{x} \in \{0, 1\}^n} \widehat{\theta}(x_1, \ldots, x_n)\cdot \theta^{\sum x_i} (1 - \theta)^{n - \sum x_i}
    \]
    Каждое слагаемое есть произведение многочлена $\theta^{\sum x_i}$ степени $\sum x_i$, многочлена $(1 - \theta)^{n - \sum x_i}$ степени $n - \sum x_i$ и какой-то константы $\widehat{\theta}(x_1, \ldots, x_n)$, что есть многочлен степени $n$. Стало быть, $\tau(\theta)$ как сумма таких слагаемых есть многочлен степени не выше $n$.
    
    Второй вопрос кажется несложным: произвольный многочлен можно однозначно разложить по $\theta^k(1-\theta)^{n-k}$, откуда можно явно построить нужную $\widehat{\theta}$. Но на самом деле ответ отрицательный~\ -- для некоторых многочленов полученные <<оценки>> не будут вообще \textit{оценками}, так как их значения могут не лежать в $\tau([0; 1])$ (помните второй пункт замечания к определению оценки?).
    
    Подтвердим вышесказанное контрпримером. Рассмотрим $\tau(x)=x(1-x)$ и двухэлементную выборку. Многочлен очевидным образом раскладывается по $x^k(1-x)^{2-k}$, поэтому если и имеется несмещённая оценка $\widehat{\theta}(X)$, то выполнено
    \[
    \left\{
    \begin{aligned}
    &\widehat{\theta}(0, 0) = 0\\
    &\widehat{\theta}(1, 1) = 0\\
    &\widehat{\theta}(1, 0) + \widehat{\theta}(0, 1) = 1.
    \end{aligned}
    \right.
    \]
    Последнее условие значит, что либо $\widehat{\theta}(1, 0) \ge 1/2$, либо $\widehat{\theta}(0, 1) \ge 1/2$. Но $\tau([0; 1]) = [0; 1/4]$, то есть значения $\widehat{\theta}$ априори не будут лежать в заданном множестве.
\end{solution}


